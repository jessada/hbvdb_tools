% $Header: /Users/joseph/Documents/LaTeX/beamer/solutions/conference-talks/conference-ornate-20min.en.tex,v 90e850259b8b 2007/01/28 20:48:30 tantau $

\documentclass{beamer}

% This file is a solution template for:

% - Talk at a conference/colloquium.
% - Talk length is about 20min.
% - Style is ornate.

\mode<presentation>
{
  \usetheme{Warsaw}
  % or ...

  \setbeamercovered{transparent}
  % or whatever (possibly just delete it)
}

\usepackage{graphics}
\usepackage{pgf,pgfarrows,pgfnodes}
\usepackage{pgfkeys}
\usepackage{tikz}
\usepackage[absolute,overlay]{textpos}
\usepackage{comment}
\usepackage{hyphenat}
\usetikzlibrary{backgrounds}
\usetikzlibrary{positioning}
\usetikzlibrary{shapes.geometric}
\usetikzlibrary{shapes.misc}
\usetikzlibrary{shapes.symbols}
\usetikzlibrary{shapes.arrows}
%\usepackage{pgfpages}
%\pgfpagesuselayout{2 on 1}[a4paper,border shrink=5mm]

\usepackage[english]{babel}
% or whatever

\usepackage[latin1]{inputenc}
% or whatever

\usepackage{times}
\usepackage[T1]{fontenc}
% Or whatever. Note that the encoding and the font should match. If T1
% does not look nice, try deleting the line with the fontenc.


\title[bvd-merge design \& architecture] % (optional, use only with long paper titles)
{HBVDB-tools}

\subtitle
{bvd-merge concept}

\author[Jessada Thutkawkorapin] % (optional, use only with lots of authors)
{Jessada Thutkawkorapin}
% - Give the names in the same order as the appear in the paper.
% - Use the \inst{?} command only if the authors have different
%   affiliation.

\institute[Science for Life Laboratory] % (optional, but mostly needed)
{Science for Life Laboratory, Stockholm}
% - Use the \inst command only if there are several affiliations.
% - Keep it simple, no one is interested in your street address.

\date[Master thesis presentation] % (optional, should be abbreviation of conference name)
{August, 2012}
% - Either use conference name or its abbreviation.
% - Not really informative to the audience, more for people (including
%   yourself) who are reading the slides online

\subject{Theoretical Computer Science}
% This is only inserted into the PDF information catalog. Can be left
% out. 



% If you have a file called "university-logo-filename.xxx", where xxx
% is a graphic format that can be processed by latex or pdflatex,
% resp., then you can add a logo as follows:

% \pgfdeclareimage[height=0.5cm]{university-logo}{university-logo-filename}
% \logo{\pgfuseimage{university-logo}}
%\pgfdeclareimage[height=0.7cm]{university-logo}{kth_svv_comp_science_comm}
%\logo{\pgfuseimage{university-logo}}



% Delete this, if you do not want the table of contents to pop up at
% the beginning of each subsection:

\AtBeginSubsection[]
{
  \begin{frame}<beamer>{Outline}
    \tableofcontents[currentsection,currentsubsection]
  \end{frame}
}


% If you wish to uncover everything in a step-wise fashion, uncomment
% the following command: 

%\beamerdefaultoverlayspecification{<+->}


\begin{document}

\begin{frame}
  \titlepage
\end{frame}

\begin{frame}{Outline}
  \tableofcontents
  % You might wish to add the option [pausesections]
\end{frame}



\tikzset{guide lines/.style={red!10, very thin}}

\tikzset{explanation box/.style={shape=rectangle, draw, red!10, minimum width=10.5cm, minimum height=1.7cm, font=\scriptsize}}
\tikzset{explanation text/.style={red, font=\small, text width=10cm}}
\tikzset{sequencing center/.style={chamfered rectangle, red!50!black!70, double=black, draw, chamfered rectangle xsep=0.2cm, chamfered rectangle ysep=0.2cm, minimum width=4cm, minimum height=2.2cm}}
\tikzset{center name text/.style={font=\scriptsize}}
\tikzset{single database/.style={shape aspect=0.08,cylinder,blue!50!black!50,shape border rotate=90,draw,bottom color=red!50!black!20,top color=white, minimum height=1.2cm, minimum width=2cm}}
\tikzset{master database/.style={shape aspect=0.08,cylinder,blue!50!black!50,shape border rotate=90,draw,font=\tiny,bottom color=red!50!black!20,top color=white, minimum width=2.4cm}}
\tikzset{slave database/.style={shape aspect=0.08,cylinder,blue!50!black!50,shape border rotate=90,draw,font=\tiny,bottom color=blue!20!black!20,top color=white, minimum width=1.4cm}}
\tikzset{master updated/.style={starburst, draw, double=black, minimum width=3.5cm, minimum height=1.9cm}}
\tikzset{slave updated/.style={starburst, draw, double=black, minimum width=2.7cm, minimum height=1.7cm}}
\tikzset{exchange arrow/.style={double arrow, draw, black, double=red!20, fill=blue!80!red!20, double arrow head indent=1ex, font=\tiny}}
\tikzset{IO entity/.style={tape, thick, draw, double=black, font=\large, minimum width=2.5cm, fill=white}}
\tikzset{IO arrow/.style={single arrow, draw, black, double=red!20, fill=blue!80!red!20, single arrow head extend=.1cm, font=\tiny, single arrow head indent=0cm }}
\tikzset{IO group/.style={cloud, draw, aspect=2, double=yellow, thin, red!100!black!100}}

\tikzset{Normal error/.style={font=\fontfamily{ppl}\fontsize{1cm}{1.2cm}\selectfont}}
\tikzset{Valid data/.style={font=\huge}}


\section{General idea}

\subsection{Overview}
\begin{frame}[fragile]
    \frametitle{Overview}
    \begin{center}
        \begin{tikzpicture}
            \draw[step=1, guide lines] (-5.3,-3.3) grid (5.3,3.3);
            \draw (0, 2.4) node [explanation box] (explanation_box){};


            \draw<1> node [below right=0.1cm of explanation_box.north west, explanation text] {\nohyphens{We will start the sample exchange scenario by letting every centers only have variant frequencies gathered by themselves.}};
            \draw<2> node [below right=0.1cm of explanation_box.north west, explanation text] {\nohyphens{First, let center A exchanges the data with center B. Center A will have a copy of database B and center will have a copy of database A.}};
                \draw<3> node [below right=0.1cm of explanation_box.north west, explanation text] {\nohyphens{Then, center A exchanges the data with center C. Center C will have a copy of databases from center A, which are databases A and B. And center A will have a copy of database C.}};
            \draw<4> node [below right=0.1cm of explanation_box.north west, explanation text] {\nohyphens{In conclusion, once the centers exchange their data, they will only make a copy of databases from other centers and keep each of them in the separate databases.}};


            
            \pgftransformshift{\pgfpoint{-2.7cm}{0.6cm}}
            \draw<1-> (0,0) node [sequencing center]{} ;
            \draw<1-> (0,0.9) node [center name text] {\bf Sequencing center A};
            \draw<1> (0,-0.1) node [single database]  {database A};
            \draw<2-> (0,0.2) node [master database]  {database A(master)};
            \draw<2-> (-0.9,-0.5) node [slave database]  {database B};
            \draw<3-> (0.9,-0.5) node [slave database]  {database C};
            \draw<2,4> (-0.9,-0.45) node [slave updated, yellow] {};
            \draw<3-> (0.9,-0.45) node [slave updated, yellow] {};



            \pgftransformshift{\pgfpoint{5.4cm}{0cm}}
            \draw<1-> (0,0) node [sequencing center] {};
            \draw<1-> (0,0.9) node [center name text] {\bf Sequencing center B};
            \draw<1> (0,-0.1) node [single database]  {database B};
            \draw<2-> (0,0.2) node [master database]  {database B(master)};
            \draw<2-> (-0.9,-0.5) node [slave database]  {database A};
            \draw<4-> (0.9,-0.5) node [slave database]  {database C};
            \draw<2,4> (-0.9,-0.45) node [slave updated, yellow] {};
            \draw<4-> (0.9,-0.45) node [slave updated, yellow] {};

            
            
            \pgftransformshift{\pgfpoint{-2.7cm}{-2.6cm}}
            \draw<1-> (0,0) node [sequencing center] {};
            \draw<1-> (0,0.9) node [center name text] {\bf Sequencing center C};
            \draw<1-2> (0,-0.1) node [single database]  {database C};
            \draw<3-> (0,0.2) node [master database]  {database C(master)};
            \draw<3-> (-0.9,-0.5) node [slave database]  {database A};
            \draw<3-> (0.9,-0.5) node [slave database]  {database B};
            \draw<3-> (-0.9,-0.45) node [slave updated, yellow] {};
            \draw<3-> (0.9,-0.45) node [slave updated, yellow] {};



            \pgftransformshift{\pgfpoint{0cm}{2.0cm}}
            \draw<3-> (-2.0,-1.1) node [exchange arrow, rotate=-50] {bvd-merge};
            \draw<4> (2.0,-1.1) node [exchange arrow, rotate=50] {bvd-merge};
            \draw<2,4> (0.0,0.6) node [exchange arrow] {bvd-merge};

        \end{tikzpicture}
    \end{center}
\end{frame}

\subsection{Read \& Write}
\begin{frame}[fragile]
    \frametitle{Read \& Write}
    \begin{center}
        \begin{tikzpicture}
            \draw[step=1, guide lines] (-5.3,-3.3) grid (5.3,3.3);
            \draw (0, 2.4) node [explanation box] (explanation_box){};


            \draw<1> node [below right=0.1cm of explanation_box.north west, explanation text, text ragged] {\nohyphens{Reading \& Writing the database will be a little complicated under this concept}};
            \draw<2> node [below right=0.1cm of explanation_box.north west, explanation text, text ragged] {\nohyphens{As it should be, the output, which is retrieved by using bvd-get, is the accumulated variant frequencies from all databases, both master and copies, in that center.}};
            \draw<3> node [below right=0.1cm of explanation_box.north west, explanation text, text ragged] {\nohyphens{But the update done by bvd-add will be done on the master database only.}};
            
            \pgftransformshift{\pgfpoint{-2.7cm}{0.6cm}}
            \draw<1-> (0,0) node [sequencing center]{} ;
            \draw<1-> (0,0.9) node [center name text] {\bf Sequencing center A};
            \draw<1-> (0,0.2) node [master database]  {database A(master)};
            \draw<1-> (-0.9,-0.5) node [slave database]  {database B};
            \draw<1-> (0.9,-0.5) node [slave database]  {database C};

            \draw<1-2> (6.0,0) node [IO entity] (output) {Output};
            \draw<1,3> (0,-3.0) node [IO entity] (input) {Input};

            \draw<1-2> (3.2,0) node [IO arrow, minimum height=2.0cm] {bvd-get};
            \draw<1,3> (0,-2) node [IO arrow, shape border rotate=90] {bvd-add};

            \draw<2> (0,-0.1) node[IO group, cloud puffs=15, minimum width=4.5cm, minimum height=2.5cm] {};
            \draw<3> (0,0.3) node[IO group, cloud puffs=15, minimum width=3.0cm, minimum height=1.7cm] {};

        \end{tikzpicture}
    \end{center}
\end{frame}


\section{Concern \& Solution}

\subsection{Mutiple exchange \& obsolete data}
\begin{frame}[fragile]
    \frametitle{Multiple exchange \& obsolete data}
    \begin{center}
        \begin{tikzpicture}
            \draw[step=1, guide lines] (-5.3,-3.3) grid (5.3,3.3);
            \draw (0, 2.4) node [explanation box] (explanation_box){};


            \draw<1> node [below right=0.1cm of explanation_box.north west, explanation text, text ragged] {\nohyphens{This problem can be easily found and can be simulated by starting from no exchange between any centers.}};
            \draw<2> node [below right=0.1cm of explanation_box.north west, explanation text, text ragged] {\nohyphens{Then center A and B exchange their data.}};
            \draw<3> node [below right=0.1cm of explanation_box.north west, explanation text, text ragged] {\nohyphens{Then another set of variant frequencies is inserted into the database of center A.}};
            \draw<4> node [below right=0.1cm of explanation_box.north west, explanation text, text ragged] {\nohyphens{And center A and C exchange their data. With this, the copy of database A in center C is more up-to-date than that in center B.}};
            \draw<5> node [below right=0.1cm of explanation_box.north west, explanation text, text ragged] {\nohyphens{Later when center B and C exchange their data, {\bf \color{black}the database A in center B or C may not be up-to-date} if it doesn't have any mechanics to prevent this problem.}};
            \draw<6> node [below right=0.1cm of explanation_box.north west, explanation text, text ragged] {\nohyphens{To prevent this, every time that the database is written, {\bf \color{black}Time Stamp}  \color{red}is recorded to the database so that it can be used later during data exchange to find the most up-to-date database.}};


            \pgftransformshift{\pgfpoint{-2.7cm}{0.6cm}}
            \draw<1-> (0,0) node [sequencing center]{} ;
            \draw<1-> (0,0.9) node [center name text] {\bf Sequencing center A};
            \draw<1> (0,-0.1) node [single database]  {database A};
            \draw<2-> (0,0.2) node [master database]  {database A(master)};
            \draw<2-> (-0.9,-0.5) node [slave database]  {database B};
            \draw<4-> (0.9,-0.5) node [slave database]  {database C};
            \draw<3-> (0,0.25) node [master updated, green] {};
            \draw<2-> (-0.9,-0.45) node [slave updated, yellow] {};
            \draw<4-> (0.9,-0.45) node [slave updated, yellow] {};

            \draw<3> (-1,-3.0) node [IO entity] (input) {Input};

            \draw<3> (-1,-2) node [IO arrow, shape border rotate=90] {bvd-add};



            \pgftransformshift{\pgfpoint{5.4cm}{0cm}}
            \draw<1-> (0,0) node [sequencing center] {};
            \draw<1-> (0,0.9) node [center name text] {\bf Sequencing center B};
            \draw<1> (0,-0.1) node [single database]  {database B};
            \draw<2-> (0,0.2) node [master database]  {database B(master)};
            \draw<2-> (-0.9,-0.5) node [slave database]  {database A};
            \draw<5-> (0.9,-0.5) node [slave database]  {database C};
            \draw<2-4> (-0.9,-0.45) node [slave updated, yellow] {};
            \draw<5> (-0.9,-0.45) node [slave updated, red] {};
            \draw<6> (-0.9,-0.45) node [slave updated, green] {};
            \draw<5-> (0.9,-0.45) node [slave updated, yellow] {};
            \draw<5> (-0.9,-0.5) node [Normal error]{?};

            
            
            \pgftransformshift{\pgfpoint{-2.7cm}{-2.6cm}}
            \draw<1-> (0,0) node [sequencing center] {};
            \draw<1-> (0,0.9) node [center name text] {\bf Sequencing center C};
            \draw<1-3> (0,-0.1) node [single database]  {database C};
            \draw<4-> (0,0.2) node [master database]  {database C(master)};
            \draw<4-> (-0.9,-0.5) node [slave database]  {database A};
            \draw<5-> (0.9,-0.5) node [slave database]  {database B};
            \draw<4,6> (-0.9,-0.45) node [slave updated, green] {};
            \draw<5> (-0.9,-0.45) node [slave updated, red] {};
            \draw<5-> (0.9,-0.45) node [slave updated, yellow] {};

            \draw<5> (-0.9,-0.5) node [Normal error]{?};

            \pgftransformshift{\pgfpoint{0cm}{2.0cm}}
            \draw<4> (-2.0,-1.1) node [exchange arrow, rotate=-50] {bvd-merge};
            \draw<5-> (2.0,-1.1) node [exchange arrow, rotate=50] {bvd-merge};
            \draw<2> (0.0,0.6) node [exchange arrow] {bvd-merge};
            
        \end{tikzpicture}
    \end{center}
\end{frame}


\subsection{Data duplication}
\begin{frame}[fragile]
    \frametitle{Data duplication \& solution}
    \begin{center}
        \begin{tikzpicture}
            \draw[step=1, guide lines] (-5.3,-3.3) grid (5.3,3.3);


            \draw (2.4,0) node [explanation box, minimum width=5.7cm, minimum height=4.5cm] (explanation_box){};
            \draw<1> node [below right=0.1cm of explanation_box.north west, explanation text, text ragged, text width=5.4cm] {\nohyphens{As bvd-add always did, it still check if inserting content is already in the system. The only difference is that it's not only check with the target database, master, but {\bf \color{black} bvd-add will check with all the databases in the system}, both master and copies, in this case, A, B and C.}};


            \pgftransformshift{\pgfpoint{-2.7cm}{0.6cm}}
            \draw<1-> (0,0) node [sequencing center]{} ;
            \draw<1-> (0,0.9) node [center name text] {\bf Sequencing center A};
            \draw<1-> (0,0.2) node [master database]  {database A(master)};
            \draw<1-> (-0.9,-0.5) node [slave database]  {database B};
            \draw<1-> (0.9,-0.5) node [slave database]  {database C};

            \draw<1-> (0,-3.0) node [IO entity] (input) {Input};

            \draw<1-> (0,-2) node [IO arrow, shape border rotate=90] {bvd-add};

            \draw<1> (0,-0.1) node[IO group, cloud puffs=15, minimum width=4.5cm, minimum height=2.5cm] {};

        \end{tikzpicture}
    \end{center}
\end{frame}


\subsection{Confidential issue}
\begin{frame}[fragile]
    \frametitle{Confidential issue}
    \begin{center}
        \begin{tikzpicture}
            \draw[step=1, guide lines] (-5.3,-3.3) grid (5.3,3.3);
            \draw (0, 2.4) node [explanation box] (explanation_box){};


            \draw<1> node [below right=0.1cm of explanation_box.north west, explanation text, text ragged] {\nohyphens{\scriptsize This database exchange is a process of copying the whole database, this mean that, {\bf \color{black}as long as there is no confidential issue in each center, there is also no confidential issue in this process.}}};




            \pgftransformshift{\pgfpoint{-2.7cm}{0.6cm}}
            \draw<1-> (0,0) node [sequencing center]{} ;
            \draw<1-> (0,0.9) node [center name text] {\bf Sequencing center A};
            \draw<1-> (0,0.2) node [master database]  {database A(master)};
            \draw<1-> (-0.9,-0.5) node [slave database]  {database B};
            \draw<1-> (0.9,-0.5) node [slave database]  {database C};

            \draw<1> (0,0.2) node [Valid data]{\color{green}\checkmark};
            \draw<1> (0,0.3) node[IO group, cloud puffs=15, minimum width=3.0cm, minimum height=1.7cm, green, double=black] {};



            \pgftransformshift{\pgfpoint{5.4cm}{0cm}}
            \draw<1-> (0,0) node [sequencing center] {};
            \draw<1-> (0,0.9) node [center name text] {\bf Sequencing center B};
            \draw<1-> (0,0.2) node [master database]  {database B(master)};
            \draw<1-> (-0.9,-0.5) node [slave database]  {database A};
            \draw<1-> (0.9,-0.5) node [slave database]  {database C};

            \draw<1> (0,0.2) node [Valid data]{\color{green}\checkmark};
            \draw<1> (0,0.3) node[IO group, cloud puffs=15, minimum width=3.0cm, minimum height=1.7cm, green, double=black] {};

            
            
            \pgftransformshift{\pgfpoint{-2.7cm}{-2.6cm}}
            \draw<1-> (0,0) node [sequencing center] {};
            \draw<1-> (0,0.9) node [center name text] {\bf Sequencing center C};
            \draw<1-> (0,0.2) node [master database]  {database C(master)};
            \draw<1-> (-0.9,-0.5) node [slave database]  {database A};
            \draw<1-> (0.9,-0.5) node [slave database]  {database B};

            \draw<1> (0,0.2) node [Valid data]{\color{green}\checkmark};
            \draw<1> (0,0.3) node[IO group, cloud puffs=15, minimum width=3.0cm, minimum height=1.7cm, green, double=black] {};

        \end{tikzpicture}
    \end{center}
\end{frame}


%\begin{frame}[fragile]
%    \frametitle{Template}
%    \begin{center}
%        \begin{tikzpicture}
%            \draw[step=1, guide lines] (-5.3,-3.3) grid (5.3,3.3);
%            \draw (0, 2.4) node [explanation box] (explanation_box){};
%
%
%            \draw<1> node [below right=0.1cm of explanation_box.north west, explanation text] {Let's start by every centers only have variant frequencies gathered by themselves.};
%
%
%            
%            \pgftransformshift{\pgfpoint{-2.7cm}{0.6cm}}
%            \draw<1-> (0,0) node [sequencing center]{} ;
%            \draw<1-> (0,0.9) node [center name text] {\bf Sequencing center A};
%            \draw<1> (0,-0.1) node [single database]  {database A};
%            \draw<2-> (0,0.2) node [master database]  {database A(master)};
%            \draw<2-> (-0.9,-0.5) node [slave database]  {database B};
%            \draw<2-> (0.9,-0.5) node [slave database]  {database C};
%            \draw<3-> (0,0.25) node [master updated, green] {};
%            \draw<3-> (-0.9,-0.45) node [slave updated, yellow] {};
%            \draw<3-> (0.9,-0.45) node [slave updated, yellow] {};
%
%
%            
%            
%            \pgftransformshift{\pgfpoint{5.4cm}{0cm}}V
%            \draw<1-> (0,0) node [sequencing center] {};
%            \draw<1-> (0,0.9) node [center name text] {\bf Sequencing center B};
%            \draw<1> (0,-0.1) node [single database]  {database B};
%            \draw<2-> (0,0.2) node [master database]  {database B(master)};
%            \draw<2-> (-0.9,-0.5) node [slave database]  {database A};
%            \draw<2-> (0.9,-0.5) node [slave database]  {database C};
%            \draw<3-> (0,0.25) node [master updated, green] {};
%            \draw<3-> (-0.9,-0.45) node [slave updated, yellow] {};
%            \draw<3-> (0.9,-0.45) node [slave updated, yellow] {};
%
%            
%            
%            \pgftransformshift{\pgfpoint{-2.7cm}{-2.6cm}}
%            \draw<1-> (0,0) node [sequencing center] {};
%            \draw<1-> (0,0.9) node [center name text] {\bf Sequencing center C};
%            \draw<1> (0,-0.1) node [single database]  {database C};
%            \draw<2-> (0,0.2) node [master database]  {database C(master)};
%            \draw<2-> (-0.9,-0.5) node [slave database]  {database A};
%            \draw<2-> (0.9,-0.5) node [slave database]  {database B};
%            \draw<3-> (0,0.25) node [master updated, green] {};
%            \draw<3-> (-0.9,-0.45) node [slave updated, yellow] {};
%            \draw<3-> (0.9,-0.45) node [slave updated, yellow] {};
%
%            \pgftransformshift{\pgfpoint{0cm}{2.0cm}}
%            \draw<2> (-2.0,-1.1) node [exchange arrow, rotate=-50] {bvd-merge};
%            \draw<2> (2.0,-1.1) node [exchange arrow, rotate=50] {bvd-merge};
%            \draw<2> (0.0,0.6) node [exchange arrow] {bvd-merge};
%
%        \end{tikzpicture}
%    \end{center}
%\end{frame}
%




    
    
\end{document}


